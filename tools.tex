android-sdk 아래 platform-tools와 tools 디렉토리가 있다.

\subsection{Hierachy Viewer}
Hierarchy는 타켓 폰에서 기본적으로는 볼 수가 없으므로, emulator를 띄우고, Hierarchy View를 활용한다.
adt가 버전업되면서 문제가 사라졌다.(확인 필요..)

\subsection{adb}
adb 에서 많이 사용하는 명령어가 adb uninstall  패키지명이다.
폰에서 프로그램을 찾아서 uninstall 하는 것보다 빠르다.
안드로이드의 특성상 여러 폰을 가지고 테스트를 하는데, 개발자 여럿이서 동일한 폰을 가지고 테스트하게 된다.
이때 다른 pc에서 올린 것을 가지고 와서 사용할 때 바로 올리진 못하고, uninstall 해야 한다.

adb uninstall com.suribada.notepad

adb install bin/*.apk 명령어는 기존에 앱이 깔려있는 경우에는 쓸 수 없다.
adb install -r [] bin/*.apk 업어쓴다.

emulator를 포함한 여러개의 device를 연결했을 때는, 
adb devices를 통해 기기를 확인할 수 있다.

adb 연결에 문제가 생겼을때 리스타트하는 방법은 다음과 같다.

\begin{verbatim}
adb -s emulator-5554 kill-server
adb -s emulator-5554 start-server
\end{verbatim}

이클립스에서 logcat 결과가 잘 안 보이는 단말기도 있다.
adb devices
adb -s 디바이스명 shell
logcat 해서 볼 수 있다.

adb -e shell(에뮬레이터)
adb -d shell(디바이스)


adb push 디렉토리 /mnt/sdcard 하면 디렉토리의 파일들이 /mnt/sdcard에 올라간다.

바로 실행하기
이게 되려면 Manifest 파일에 등록되어 있어야만 한다.
adb shell am start -n com.nhn.android.search/com.nhn.android.alarmclock.AlarmClock
%http://www.xinotes.org/notes/note/1441/
%http://en.androidwiki.com/wiki/ADB_Shell_Command_Reference 참고하자!
Intent Filter가 없는 경우에는 android:exported="true"가 있어야만, 위 명령어를 쓸 수가 있다.
아니면, java.lang.SecurityException을 발생시킨다.

%http://developer.android.com/guide/topics/manifest/activity-element.html#exported

Shell에 들어가면 /system/bin 아래 사용가능한 메소드들이 보인다. 실제 디바이스에서는 잘 쓰지 못한다.

\section{dumpsys}
service list 하면 나오는 것이 dumpsys 목록들이다.

\section{DDMS}
unKnownHost 문제가 자꾸 나면 emulator를 재시작해보도록 한다.
unknownHostException 나는 경우는 Internet 퍼미션 없을때도 생겨난다.

System.out.println으로 출력하면 Logcat에서는 Tag에 system.out 으로 표시된다.
System.err.println은system.err로 표시된다.

\section{monkey}
크래쉬나, Exception, ANR을 만나면, Monkey가 끝나고 error를 리포트 한다.
% http://blog.daum.net/whisperlip/7287317
파일로 결과를 확인하도록 하자.
\begin{verbatim}
monkey -p com.suribada.notepad -s 269 --pct-touch 50 --pct-nav 5 
--pct-majornav 25 --pct-appswitch 10 --pct-anyevent 10 -v 20000 > /mnt/sdcard/monkey.log
\end{verbatim}

